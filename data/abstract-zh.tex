\chapter{摘要}

计算机辅助艺术设计例如计算机绘画、计算机自动上色等是一个科研人员长期进行探索的问题, 因为其与人类多样化的表达方式相关,而人类感受与复杂多变, 使其成为一个挑战性强而又持续探索的领域。 随着人工智能与机器学习最近几年的发展, 计算机艺术辅助系统有了新的突破, 例如GAN能够自动生产图片, pixel2pixel能够自动依据已有风格进行色彩填充,PaintsChainer能在标注颜色的情况下为二次元图片自动上色。 

本文提出了一种基于语义智能理解的自动配色和上色系统。 该系统主要解决了目前自动上色系统只能通过标注色彩、有限类型进行色彩填充或依赖已有色彩风格进行迁移, 从而使得系统更加智能,可以直接对人类的语言进行响应。对于设计师而言, 本系统可以将文本的描述, 即时转化为配色方案对设计线稿进行填充,大大方便了其使用便捷性。 而配色方案来自于艺术作品,有独特的审美性可以帮助到设计师。

本文为解决这一问题使用了3个创新点: 1. 使用了深度学习模型对语义进行理解; 2. 将图片信息与语义层面的隐藏信息进行关联, 其图片风格相似性的偏序特性被保持到其语义描述的偏序特性中; 3. 本文使用了泛洪填充算法(Flood Fill Algorithm)与随机游走(Random Walk)的方式, 对图像进行自动填充; 

文本解决的难点一共有4点: 1. 使用word2vect对输入的文字、艺术品的文本信息进行语义理解,设计带有语义信息的编码形式; 2. 输入文字的编码信息与艺术品文本的编码信息对比,需要搜索其中距离最近者; 3. 艺术品的色彩特征提取; 4. 上色出现的一系列问题:色彩连续性、色彩比例、效果图的光影效果。

此算法在室内设计线稿图纸的实际测试中取得了良好的结果。达到了目前业界的最好效果。 


{
    \vspace{1em}
    \setlength{\parindent}{0em}
    \textbf{关键词} \; 计算机辅助设计 \; 深度学习\; 表示学习\; 自然语言理解 \par
}
